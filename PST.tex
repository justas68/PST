\documentclass{VUMIFPSkursinis}
\usepackage{algorithmicx}
\usepackage{algorithm}
\usepackage{algpseudocode}
\usepackage{amsfonts}
\usepackage{float}
\usepackage{amsmath}
\usepackage{bm}
\usepackage{caption}
\usepackage{color}
\usepackage{float}
\usepackage{graphicx}
\usepackage{listings}
\usepackage{subfig}
\usepackage{ltablex}
\usepackage{longtable}
\usepackage{wrapfig}
\usepackage{enumitem}
\usepackage{subfig}
\usepackage{pbox}
\renewcommand{\labelenumii}{\theenumii}
\renewcommand{\theenumii}{\theenumi.\arabic{enumii}.}
\renewcommand{\labelenumiii}{\theenumiii}
\renewcommand{\theenumiii}{\theenumii\arabic{enumiii}.}
% Titulinio aprašas
\university{Vilniaus universitetas}
\faculty{Matematikos ir informatikos fakultetas}
\department{Programų sistemų katedra}
\papertype{Projektinis darbas}
\title{Internetinės parduotuvės Kilobaitas testavimas}
\titleineng{Web shop Kilobaitas testing}
\status{3 kurso 3 grupės studentas}
\author{Justas Tvarijonas}  
\supervisor{Vytautas Valaitis, Asist., Dr.}
\date{Vilnius – \the\year}

\begin{document}
\maketitle
\sectionnonum{Anotacija}
Šiuo darbu yra siekiama sukurti testavimo planą ir atlikti testavimą www.kilobaitas.lt svetainei skirtai kompiuterinės technikos pardavimui.
\begin{itemize}
	\item Justas Tvarijonas - tvarijonasjustas@gmail.com
\end{itemize}
\tableofcontents
\sectionnonum{Testavimo planas}
\section{Testavimo plano identifikatoriai}
KBT01.XX, kur XX - testo numeris.
\vspace{1cm}
\section{Nuorodos}
\begin{itemize}
	\item http://pst.valaitis.net/PST
	\item https://jmpovedar.files.wordpress.com/2014/03/ieee-829.pdf
\end{itemize}
\vspace{1cm}
\section{Įvadas}
Šis testavimo planas yra skirtas internetinės parduotuvės Kilobaitas testavimui. Pagrindinis testavimo tikslas įsitikinti ar su skirtingais duomenimis internetinė svetainė veikia korektiškai. Bus testuojama prekių pridėjimas į krepšelį, prekių palyginimas, apmokėjimas. Visas testavimas bus vykdomas rankiniu būdu.
\vspace{1cm}
\section{Testavimo objektas}
www.kilobaitas.lt
\vspace{1cm}
\section{Funkcionalumas, kurį testuosime}
\begin{itemize}
	\item Prekių pridėjimas į krepšelį.
	\item Prekių palyginimas.
	\item Mokėjimo suformulavimas. 
	\item Prekių paieška.
\end{itemize}
\vspace{1cm}
\section{Funkcionalumas, kurio netestuosime}
\begin{itemize}
	\item Apmokėjimas - nėra galimybės jo imituoti, o pačiam leisti pinigų nesinori.
	\item Profilio funkcionalumas.
\end{itemize}
\vspace{1cm}
\section{Strategija}
\begin{itemize}
	\item Testavimui naudojamas Asus nešiojamas kompiuteris su Google Chrome interneto naršykle.
	\item Metrikos
	\begin{itemize}
		\item Surastų klaidų skaičius.
		\item Teisingų testų skaičius.
		\item Testavimui skirtas laikas.
	\end{itemize}
	\item Regresiniai testai - nėra.
	\item Testuojama pradinė konfiguracija.
	\item Specialūs įrankiai - nėra.
	\item Apmokymas naudotis įrankiais - nėra.
\end{itemize}
\vspace{1cm}
\section{Sekmės ir nesekmės kriterijai}
Testą laikome pasisekusiu, jeigu atlikus atitinkamą veiksmą programa pateiks tikėtą rezultatą ir nerodys klaidos pranešimų. Nepeasisekusiu laikome tokį testą, kurį atlikus gausime nenumatytą išvestį.
\vspace{1cm}
\section{Testo nutraukimo ar sustabdymo kriterijai}
\begin{itemize}
	\item Testą nutraukiame tuo atvėju, jeigu naujas atliekas testas mum daugiau informacijos suteikti nebegali, tokiu atvėju testavimą nutrauksime įvertinti savo testavimo atvejus tolimesniam testavimui.
	\item Jeigu svetainės palaikytojas nutraukia savo veiklą, lauksime, kol internetinė parduotuvė vėl bus paleista.
\end{itemize}

\vspace{1cm}
\section{Testų rezultatai}
\begin{itemize}
	\item Testų planas.
	\item Testų scenarijai.
	\item Klaidų sąrašas.
	\item Metrikos.
	\item Testavimo strategija.
	\item Testavimo išvada.
\end{itemize}
\vspace{1cm}
\section{Likusios užduotys}

\begin{center}
\begin{tabular}{ |c|c|c| } 
 \hline
 Užduotis & Vykdytojas & Būsena \\ \hline
 Sukurti testavimo planą. & Justas Tvarijonas. & Pabaigta \\ \hline
 Prekių pridėjimas į krepšelį.  & Justas Tvarijonas. & Pabaigta. \\ \hline
 Kelių prekių palyginimas. & Justas Tvarijonas. & Pabaigta. \\ \hline
 Mokėjimo suformulavimas & Justas Tvarijonas. & Pabaigta. \\ \hline
 Prekių paieška. & Justas Tvarijonas. & Pabaigta. \\ \hline
\end{tabular}
\end{center}
\vspace{1cm}
\section{Aplinkos reikalavimai}
\begin{itemize}
	\item Reikalinga prieeiga prie interneto ryšio.
	\item Reikalinga naujausios versijos interneto naršyklė.
	\item Turi veikti "Kilobaitas" elektroninė parduotuvė.
\end{itemize}
\vspace{1cm}
\section{Personalas ir reikalingi apmokymai}
Šiam testavimui reikėtų bent 1 pilnu etatu dirbančio testavimo profesionalo, kuris labiau įsigilinęs ir svetainės veikimą galėtų pastebėti daugiau trūkumų(atliktų ad hoc testavimą). Toks testuotojas sistemą ištestuotų per savaitę laiko, tačiau, dėl resursų trūkumo testavimas bus atliekamas programų sistomų studento.
\vspace{1cm}
\section{Atsakomybės}
\begin{center}
\begin{tabular}{ |c|c| } 
 \hline
 Atsakomybė & Justas Tvarijonas \\ \hline
 Rankinių testų kūrimas. & X \\ \hline
 Rankinių testų vykdymas. & X \\ \hline
 Testavimo plano sudarymas. & X \\ \hline
\end{tabular}
\end{center}
\vspace{1cm}
\section{Tvarkaraštis}
Testavimo plano ir testavimo atvejų sudarymas ir testų atlikimas bus įvykdytas per 1-2 dienas.\newline

Darbas bus atliekamas tokiais žingsniais:
\begin{enumerate}
	\item{Testavimo plano sudarymas.}
	\item{Testavimo atvejų dokumento sudarymas.}
	\item{Testavimo atvejų sudarymas.}
	\item{Testų vykdymas.}
	\item{Rezultatų ir išvadų sudarymas.}
\end{enumerate}
\vspace{1cm}
\section{Rizikos}
\begin{itemize}
	\item Testavimas gali būti nesėkmingas dėl laiko stokos.
	\item Dėl dokemtacijos trūkumo testai gali būti ne pilni, gali būti nepadengti visi testavimo atvėjai.
	\item Dėl žmonių trūkumo gali būti nepadengti visi atvėjai.
	\item Dėl programinės įrangos trūkumo gali būti nepadengti sudėtingesni use cases.
	\item 
\end{itemize}
\sectionnonum{Testavimo atvejai}
\section{Įvadas}
Šioje dalyje bus pateikti Kilobaitas elektroninės parduotuvės testavimo scenarijai. Šiuo darnu siekiama įsitikinti, kad pagrindiniai svetainės funkcionalumai veikia korektiškai. Bus testuojamos šios dalys:
\begin{enumerate}
	\item Prekių pridėjimas į krepšelį.
	\item Kelių prekių palyginimas.
	\item Mokėjimo suformulavimas.
	\item Prekių paieška.
\end{enumerate}
\section{Testavimo atvejai}
\begin{center}
	\begin{tabular}{ |p{5cm}|p{10cm}|}
	\hline
	Testavimo atvejo numeris & KBT01.00. \\ \hline
	Testavimo atvejis &  Prekių pridėjimas į krepšelį.\\ \hline
	Testavimo atvejo aprašymas & Patikrinti ar pridedant papildomą prekę(prekes), jos kiekis teisingai atvaizduojamas prekių krepšelyje.  \\ \hline
	Testavimo eiga & 
	\begin{enumerate} 
		\item Susirandema norimą pridėti prekę.
		\item Įvedame 1 į prekės kiekio laukelį laukelį.
		\item Spaudžiame "Pridėti į krepšelį".
	\end{enumerate} \\ \hline
	Tikėtasis rezultatas & Pridėjus naują prekę tai atsispinti krepšelyje. pagrindiniame lange matomoje krepšelio piktogramoje jame esančių prekių skaičius padidėjo 1, atsidarius krepšelį matome šią prekę.  \\ \hline
	Gautas rezultatas & Prekė pridėta korektiškai. \\ \hline
	Statusas & Atlikta. \\ \hline
	\end{tabular}
\captionof{table}{Prekių pridėjimas į krepšelį} 
\vspace{1cm}
\begin{tabular}{ |p{5cm}|p{10cm}|}
	\hline
	Testavimo atvejo numeris & KBT01.01. \\ \hline
	Testavimo atvejis & Prekės kiekio papildymas. \\ \hline
	Testavimo atvejo aprašymas & Patikrinti ar papildant jau pridėtos prekės kiekį, jis pasikeičia tinkamai. \\ \hline
	Testavimo eiga & 
	\begin{enumerate} 
		\item Susirandame prekę, kurią jau turime krepšelyje.
		\item Įvedame 1 į prekės kiekio laukelį laukelį.
		\item Spaudžiame "Pridėti į krepšelį".
	\end{enumerate} \\ \hline
	Tikėtasis rezultatas & Pridėjus naują prekę tai atsispinti krepšelyje. pagrindiniame lange matomoje krepšelio piktogramoje jame esančių prekių skaičius padidėjo 1, atsidarius krepšelį matome, kad prie buvusio šios prekės kiekio pridėta 1. \\ \hline
	Gautas Rezultatas & Prekės kiekis buvo papildytas korektiškai. \\ \hline
	Statusas & Atlikta. \\ \hline
	\end{tabular}
\captionof{table}{Prekės kiekio papildymas.}
\vspace{1cm}
\begin{tabular}{ |p{5cm}|p{10cm}|}
	\hline
	Testavimo atvejo numeris & KBT01.02. \\ \hline
	Testavimo atvejis & Prekės pridėjimas pateikiant ne skaitinę reikšmę.\\ \hline
	Testavimo atvejo aprašymas & Patikrinti ar svetainė tinkamai apdoroja neskaitines reikšmes įvestas į kiekio laukelį. \\ \hline
	Testavimo eiga & 
	\begin{enumerate} 
		\item Susirandema norimą pridėti prekę.
		\item Įvedame C į prekės kiekio laukelį laukelį.
		\item Spaudžiame "Pridėti į krepšelį".
	\end{enumerate} \\ \hline
	Tikėtasis rezultatas & Klaidos pranešimas/nepapildytas krepšelis. \\ \hline
	Gautas Rezultatas & Į krepšelį pridėta pasirinktos prekės 1 vienetas. \\ \hline
	Statusas & Nepavyko. \\ \hline
	\end{tabular}
\captionof{table}{Prekės pridėjimas pateikiant ne skaitinę reikšmę.}
\vspace{1cm}
\begin{tabular}{ |p{5cm}|p{10cm}|}
	\hline
	Testavimo atvejo numeris & KBT01.03. \\ \hline
	Testavimo atvejis & Prekės pridėjimas pateikiant neigiamą reikšmę. \\ \hline
	Testavimo atvejo aprašymas & Patikrinti ar svetainė tinkamai apdoroją į prekės kiekio lauką įvedus neigiamą reikšmę. \\ \hline
	Testavimo eiga &
	\begin{enumerate} 
		\item Susirandema norimą pridėti prekę.
		\item Įvedame -5 į prekės kiekio laukelį laukelį.
		\item Spaudžiame "Pridėti į krepšelį".
	\end{enumerate} \\ \hline
	Tikėtasis rezultatas & Klaidos pranešimas/nepapildytas krepšelis. \\ \hline
	Gautas Rezultatas & Į krepšelį pridėtos -5 prekės. \\ \hline
	Statusas & Nepavyko. \\ \hline
	\end{tabular}
\captionof{table}{Prekės pridėjimas pateikiant neigiamą reikšmę.}
\vspace{1cm}
\begin{tabular}{ |p{5cm}|p{10cm}|}
	\hline
	Testavimo atvejo numeris & KBT01.04. \\ \hline
	Testavimo atvejis & 3 prekių palyginimas. \\ \hline
	Testavimo atvejo aprašymas & Patikrinti ar svetainėje veikia tinkamas prekių palyginimas, lyginant nedidelį kiekį prekių. \\ \hline
	Testavimo eiga & 
	\begin{enumerate} 
		\item Susirandema norimas 3 prekes.
		\item Prie kiekvienos spaudžiame "palyginti" varnelę.
		\item Spudžiame mygtuką "Palyginimas".
	\end{enumerate} \\ \hline
	Tikėtasis rezultatas & Matome visas pasirinktas prekes su jų psecifikacijomis palyginimo lange. \\ \hline
	Gautas Rezultatas & Prekių palyginimas tinkamai atvaizduotas. \\ \hline
	Statusas & Atlikta. \\ \hline
	\end{tabular}
\captionof{table}{3 prekių palyginimas.}
\vspace{1cm}
\begin{tabular}{ |p{5cm}|p{10cm}|}
	\hline
	Testavimo atvejo numeris & KBT01.05. \\ \hline
	Testavimo atvejis & 1 prekės palyginimas. \\ \hline
	Testavimo atvejo aprašymas & Patikrinti ar svetainė tinkamai veikia prie lyginamų prekių pridėjus tik 1 prekę. \\ \hline
	Testavimo eiga &  
	\begin{enumerate} 
		\item Susirandema norimą.
		\item Spaudžiame "palyginti" varnelę.
		\item Spudžiame mygtuką "Palyginimas".
	\end{enumerate} \\ \hline
	Tikėtasis rezultatas & Parodomas pranešimas, kad lyginimui reikia bent 2 prekių.  \\ \hline
	Gautas Rezultatas & Atsidaro langas, kuriame rodoma vienintelė pasirinkta prekė. \\ \hline
	Statusas & Nepavyko. \\ \hline
	\end{tabular}
\captionof{table}{1 prekės palyginimas.}
\vspace{1cm}
\begin{tabular}{ |p{5cm}|p{10cm}|}
	\hline
	Testavimo atvejo numeris & KBT01.06. \\ \hline
	Testavimo atvejis & 20 prekių palyginimas. \\ \hline
	Testavimo atvejo aprašymas & Patikrinti ar svetainė tinkamai atvaizduoja didelio prekių kiekio palyginimą. \\ \hline
	Testavimo eiga &  
	\begin{enumerate} 
		\item Susirandema norimas 20 prekių.
		\item Prie kiekvienos spaudžiame "palyginti" varnelę.
		\item Spudžiame mygtuką "Palyginimas".
	\end{enumerate} \\ \hline
	Tikėtasis rezultatas & Prekės atvaizduojamos, bei navigacijos pagalba galima matyti jų specifikacijų skirtumus. \\ \hline
	Gautas Rezultatas & Prekių palyginimas tinkamai atvaizduotas. \\ \hline
	Statusas & Atlikta. \\ \hline
	\end{tabular}
\captionof{table}{20 prekių palyginimas.}
\vspace{1cm}
\begin{tabular}{ |p{5cm}|p{10cm}|}
	\hline
	Testavimo atvejo numeris & KBT01.07. \\ \hline
	Testavimo atvejis & Mokėjimo suformulavimas su teigiama suma krepšelyje. \\ \hline
	Testavimo atvejo aprašymas & Patikrinti ar tinkamai suformuojama mokėjimo sąskaita, kai krepšelio vertė yra teikiamas skaičius. \\ \hline
	Testavimo eiga &
	\begin{enumerate} 
		\item susirandame norimą prekę.
		\item Į kiekio laukelį įvedame 1.
		\item Pridedame į krepšelį.
		\item Atsidarome krepšelį.
		\item Spaudžiame apmokėti.
		\item Suvedame duomenis.
		\item Spaudžiame "Tęsti".
	\end{enumerate} \\ \hline
	Tikėtasis rezultatas & Gauta apmokėjimo sąskaita su mokėjimo pasirinkimais. \\ \hline
	Gautas Rezultatas & Gauta apmokėjimo sąskaita su mokėjimo pasirinkimais. \\ \hline
	Statusas & Atlikta. \\ \hline
	\end{tabular}
\captionof{table}{Mokėjimo suformulavimas su teigiama suma krepšelyje.}
\vspace{1cm}
\begin{tabular}{ |p{5cm}|p{10cm}|}
	\hline
	Testavimo atvejo numeris & KBT01.08. \\ \hline
	Testavimo atvejis & Mokėjimo suformulavimas su neigiama suma krepšelyje. \\ \hline
	Testavimo atvejo aprašymas & Patikrinti ar svetainė tinkamai reaguoja, kai krepšelio vertė yra neigiamas skaičius. \\ \hline
	Testavimo eiga & 
	\begin{enumerate}
	\item susirandame norimą prekę.
	\item Į kiekio laukelį įvedame -1.
	\item Pridedame į krepšelį.
	\item Atsidarome krepšelį.
	\item Spaudžiame apmokėti.
	\item Suvedame duomenis.
	\item Spaudžiame "Tęsti".
\end{enumerate} \\ \hline
	Tikėtasis rezultatas & Išvestas pranešimas, kad krepšelio suma negali būti neigiama. \\ \hline
	Gautas Rezultatas & Sistemo klaida. \\ \hline
	Statusas & Nepavyko. \\ \hline
	\end{tabular}
\captionof{table}{Mokėjimo suformulavimas su neigiama suma krepšelyje.}
\vspace{1cm}
\begin{tabular}{ |p{5cm}|p{10cm}|}
	\hline
	Testavimo atvejo numeris & KBT01.09. \\ \hline
	Testavimo atvejis & Prekių paieška su gamintojo pavadinimu. \\ \hline
	Testavimo atvejo aprašymas & Patikrinti ar svetainė tinkamai parodo rezultatus ieškant pagal gamintojo pavadinimą.\\ \hline
	Testavimo eiga & 
	\begin{enumerate}
	\item Į paieškos laukelį įvedame "Samsung".
	\item Spaudžiame ieškoti.
\end{enumerate} \\ \hline
	Tikėtasis rezultatas & Prekių sąrašas, kurių gamintojas yra Samsung. \\ \hline
	Gautas Rezultatas & prekių, kurių gamintojas Samsung sąrašas.\\ \hline
	Statusas & Atlikta. \\ \hline
	\end{tabular}
\captionof{table}{Prekių paieška su gamintojo pavadinimu.}
\vspace{1cm}
\begin{tabular}{ |p{5cm}|p{10cm}|}
	\hline
	Testavimo atvejo numeris & KBT01.10. \\ \hline
	Testavimo atvejis & Paieška su tuščiu paieškos laukeliu. \\ \hline
	Testavimo atvejo aprašymas & Patikrinti ar svetainė tinkamai reaguoja į tuščią paiešką. \\ \hline
	Testavimo eiga &
	\begin{enumerate}
	\item Įsitikiname, kad paieškos laukelis yra tuščias.
	\item Spaudžiame ieškoti.
	\end{enumerate} \\ \hline
	Tikėtasis rezultatas & Klaidos pranešimas, pranešantis, kad paieška negali būti atliekama neįvadus raktinių simbolių. \\ \hline
	Gautas Rezultatas & Klaidos pranešimas, dėl to, kad paieškai reikia būti įvedus bent 3 simbolius. \\ \hline
	Statusas & Atlikta \\ \hline
	\end{tabular}
\captionof{table}{Paieška su tuščiu paieškos laukeliu.}
\vspace{1cm}
\begin{tabular}{ |p{5cm}|p{10cm}|}
	\hline
	Testavimo atvejo numeris & KBT01.11 \\ \hline
	Testavimo atvejis & Paieška į laukelį įvedant html sintaksės tekstą. \\ \hline
	Testavimo atvejo aprašymas & Patikrinti ar svetainė neužlūžta įvedus html kodą į paieškos laukelį. \\ \hline
	Testavimo eiga & 
	\begin{enumerate}
	\item Įvedame "<h1>test</h1>" į paieškos laukelį.
	\item Spaudžiame ieškoti.
	\end{enumerate} \\ \hline
	Tikėtasis rezultatas & paieška su tuščių rezultatu. \\ \hline
	Gautas Rezultatas & Systemos klaida. \\ \hline
	Statusas & Nepavyko. \\ \hline
	\end{tabular}
\captionof{table}{Paieška į laukelį įvedant html sintaksės tekstą.}
\vspace{1cm}
\end{center}
\newpage
\section{Defektų aprašas}
\begin{center}
\begin{tabular}{ |p{1.8cm}| p{4cm} | p{7cm} | p{2cm} |}
	\hline
	Nr &  Funkcionalumas & Aprašymas & Sunkumas \\ \hline
	KBT01.02 & Prekės pridėjimas pateikiant ne skaitinę reikšmę. & Bandant pridėti neskaitinę reikšmę elektroninė parduotuvė automatiškai prideda prekę su kiekiu 1. & Lengvas. \\ \hline
	KBT01.03 & Prekės pridėjimas pateikiant neigiamą reikšmę. & Bandant pridėti neigiamą reikšmę svetainė neparodo jokio klaidos pranešimo, ir atlieką šį veikstą, taip krepšelio galututinė suma gali būti neigiama. & Vidutinis. \\ \hline
	KBT01.05 & 1 prekės palyginimas. & Bandant palyginti 1 prekę svetainė neišveda jokio klaidos pranešimo, o atidaro naują langą su tu viena preke. & Lengvas. \\ \hline
	KBT01.08 & Mokėjimo suformulavimas su neigiama suma krepšelyje. & Į krepšelį pridėjus neigiamą kiekį prekių ir tęsiant mokėjimą gaunama sistemo klaida. & Lengvas. \\ \hline
	KBT01.11 & Paieška į laukelį įvedant html sintaksės tekstą. & Bet koks HTML sintakse įvestas tekstas į paieškos laukelį parodo sistemos klaidą. & Lengvas. \\ \hline
\end{tabular}
\captionof{table}{Defektų sąrašas.}
\end{center}
\section{Išvada}
Atliekant rankinį testavimą rastą klaidų, kai įvedamos reikšmės nėra tipinės, kasdienis vartotojas jas įvestų nebent netyčia. Rastos klaidos nėra labai reikšmingos, kadangi jos nekeičia vartotojo patirties, netrigdo funkcionalumo. Visas rastas klaidas galima priskirti tai pačiai kategorijai, kadangi jos visos kyla iš ribinės(netipinės) įvesties. Tam tikros kalidos yra paprastesnės, kai nėra išeinama iš pagrindinių langų, tačiau tam tikrose klaidose yra parodomas sistemos klaidos pranešimas, iš kurio nėra galimybės grįžti atgal į pagrindinį puslapį. Tačiau Kilobaitas elektroninė parduotuvė yra pakankamai stabili didžiajai daliai vartotojų.
\end{document}