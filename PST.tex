\documentclass{VUMIFPSkursinis}
\usepackage{algorithmicx}
\usepackage{algorithm}
\usepackage{algpseudocode}
\usepackage{amsfonts}
\usepackage{float}
\usepackage{amsmath}
\usepackage{bm}
\usepackage{caption}
\usepackage{color}
\usepackage{float}
\usepackage{graphicx}
\usepackage{listings}
\usepackage{subfig}
\usepackage{ltablex}
\usepackage{longtable}
\usepackage{wrapfig}
\usepackage{enumitem}
\usepackage{subfig}
\usepackage{pbox}
\renewcommand{\labelenumii}{\theenumii}
\renewcommand{\theenumii}{\theenumi.\arabic{enumii}.}
\renewcommand{\labelenumiii}{\theenumiii}
\renewcommand{\theenumiii}{\theenumii\arabic{enumiii}.}
% Titulinio aprašas
\university{Vilniaus universitetas}
\faculty{Matematikos ir informatikos fakultetas}
\department{Programų sistemų katedra}
\papertype{Projektinis darbas}
\title{Internetinės aplikacijos .. testavimas}
\titleineng{Web aplication .. testing}
\status{3 kurso 3 grupės studentas}
\author{Justas Tvarijonas}
\secondauthor{Džiugas Mažulis}    
\supervisor{Vytautas Valaitis, Asist., Dr.}
\date{Vilnius – \the\year}

\begin{document}
\maketitle
\sectionnonum{Anotacija}
\tableofcontents
\sectionnonum{Testavimo planas}
\section{Testavimo plano identifikatoriai}
\section{Nuorodos}
\section{Įvadas}
\section{Testavimo objektas}
\section{Rizikos}
\section{Funkcionalusmas, kurį testuosime}
\section{Funkcionalunas, kurio netestuosime}
\section{Strategija}
\section{Sekmės ir nesekmės kriterijai}
\section{Testo nutraukimo kriterijai}
\section{Testų planai}
\section{Likusios užduotys}
\section{Aplinkos reikalavimai}
\section{Personalas ir reikalingi apmokymai}
\section{Atsakomybės}
\section{Rizikos}
\sectionnonum{Testavimo atvejai}
\section{Įvadas}
\section{Testavimo atvejai}
\section{Defektų aprašas}
\section{Išvada}
\end{document}